\documentclass[oneside]{article}
\usepackage{listings}
\usepackage[hmargin=1.5in,vmargin=1.0in]{geometry}
\usepackage{graphicx}
\usepackage[T1]{fontenc}
\begin{document}
\lstset{language=C, 
        frame=single, 
        breaklines=true,
        basicstyle=\small\ttfamily,
        columns=fullflexible}
\section*{ECE 354: Part 3}
Group: ECE.354.S11-G031 \\
Members: Ben Ridder (brridder), Casey Banner (cccbanne), 
David Janssen (dajjanss) \\ \\
This document is concerned with the high level implementation of the 
specified primitives. As such, most of the low level back end is omitted for
simplicity. Any kernel structures required to understand the primitive are
described when the primitive is discussed.

\subsection*{Design of Delayed Send} 
The \texttt{delayed\_send(...)} primitive works using a separate queue that holds
all delayed messages regardless of which process they belong to. Inside the 
queue they are ordered with the messages that are to be sent the soonest at the
front of the queue. This function takes three parameters: a process id of the 
process that is being sent a message, a message envelope containing the message, and
the delay value in milliseconds. A system is call made causing a switch to kernel mode. 
After the switch the message the parameters are checked to make sure there is a valid 
receiver id and delay value, invalid parameters return an \texttt{RTX\_ERROR}. If this 
check passes then the message is inserted into the correct spot in the delayed messages 
queue where it stays until it is forwarded by the timer into the corresponding message 
queue. The pseudo code for the delayed send is in Listing~\ref{delay}. The pseudo code 
and implementation information for the timer forwarding is shown below in the Timer 
I-Process section.

\lstset{caption={Pseudo code for delayed send },label=delay}
\begin{lstlisting}
int delayed_send(int processs_id, void* message_envelope, int delay){
    switch to kernel mode;

    if (receiver_id is invalid || delay is invalid){
        return RTX_ERROR;
    } else{
        set message sender id;
        set message receiver id;
        set delay information;
        insert into delay queue;

        return RTX_SUCCESS;
    }
}
\end{lstlisting}

\subsection*{The KCD Process}
The Keyboard Command Decoder is implemented in \texttt{system\_processes.c}. The 
\texttt{process\_kcd()} blocks on receiving messages. Once a message is received the 
message type is checked to see if it is either key input or command registration if it 
is neither the message is released. If the type is key input and the message body matches 
to a previously registered command then the message data is sent to that process and the 
original message is sent to the CRT. If the message body is not previously registered it 
is not sent anywhere and again the original message is sent to the CRT. If the type is 
command registration it then registers a command for the sender pid where the command is 
is the message body.The pseudo code for the KCD can been seen in Listing~\ref{kcd}. 

\lstset{caption={Pseudo code for implementing the KCD Processes}, label=kcd}
\begin{lstlisting}
void process_kcd(){
    initialize local variables;

    while(1){
        receive a message;
        if (message type = key input){
            if (message body = registered command){
                copy message and send to process;
            }
        
            send message to CRT;    

        } else if (message type = command registration){
            register a new command;
        } else {
            release message;
        }
    }
}
\end{lstlisting}

\subsection*{The CRT Process}
The CRT process is also implemented in \texttt{system\_process.c}. Much like the KCD 
the \texttt{process\_crt\_display()} blocks until a message is received. Once a message 
has been received the type is checked. If the type is not output or key input then 
the process and message are released. If type is correct then the message data is sent to 
the uart and the process and message are released. Pseudo code for the process can be found 
in Listing~\ref{crt}.

\lstset{caption={Pseudo code for the CRT process}, label=crt}
\begin{lstlisting}
void process_crt_display(){
    initialize local variables;

    while(1){
        receive a message;
        if (message type = output || message type = key input){
                send message data to uart;
        }
        
        release message;
        release process;
    }
}
\end{lstlisting}

\subsection*{The UART I-Process}
The UART interrupt process can be found in \texttt{system\_process.c}. This process runs with 
interrupts. After the interrupt has fired the UART status register is checked to 
determine whether a read of a write is to occur. When reading in data all characters are appended 
to a buffer until a carriage return is read. At this point a newline and null value are appended 
and a message is created. A check is made to determine the first value in the string buffer. If 
it is a percent (\%) the message is sent to the KCD, if it is an exclamation mark (!) the message 
is sent to the CRT and the string buffer is sent to the UART debug decoder, otherwise the message 
is sent to the CRT. After this the interrupts are re-enabled. If it is a write state, a character is 
written to serial port and tx ready is masked. Then the process is released.
We made a UART I-Process. Pseudo code here at Listing~\ref{uart}.

\lstset{caption={Pseudo code for the UART I-Process}, label=uart}
\begin{lstlisting}
void i_process_uart(){
    initialize local variables;

    while(1){
        determine uart state;

        if(state = read){
            read characater;
            append character to buffer;
            if (character = carriage return){
                append newline to buffer;
                append null to buffer;
                
                create a new message;
                set message data to buffer;

                if(buffer[0] = '%'){
                    send message to KCD;                
                } else if (buffer[0] = '!'){
                    send message to CRT;
                    send buffer to uart debug decoder;
                } else {
                    send message to CRT;
                }
            }
                enable interrupts;
        } else if (state = write){
            write out character;
            mask tx ready;
        }
    
        release processor;
    }
}
\end{lstlisting}

\subsection*{The Timer I-Process}

We made a Timer I-Process. Pseudo code here at Listing~\ref{timer}.

\lstset{caption={Pseudo code for the UART I-Process}, label=timer}
\begin{lstlisting}
UART_Process();
\end{lstlisting}

\subsection*{The Wall Clock Process}

There is a clock on the wall. Pseudo code here at Listing~\ref{wallclock}.

\lstset{caption={Pseudo code for the wall clock process}, label=wallclock}
\begin{lstlisting}
wall_clock_Process();
\end{lstlisting}

\end{document}
