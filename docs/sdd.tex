\documentclass[oneside]{report}
\usepackage{listings}
\usepackage[hmargin=1.4in,vmargin=1.0in]{geometry}
\usepackage[T1]{fontenc}
\usepackage[latin1]{inputenc}
\usepackage{float}
\begin{document}
\lstset{language=C, 
        frame=single, 
        breaklines=true,
        basicstyle=\small\ttfamily,
        columns=fullflexible}

% Pre-amble - title

\title{ECE354: RTX Project Final Report}
\author{Ben Ridder - brridder \\
Casey Banner - cccbanne \\
David Janssen - dajjanss }
\date{\today}

\maketitle

\tableofcontents

\chapter{\textsc{Software Design}}

\section{\textsc{Introduction}}

\section{\textsc{Global Information}}
% Data structures (queues, PCB, message formats, tracebuffer, etc)
% Variables
% Constants (PIDS, msgID, return codes, status codes, msg types, etc)
% Stacks
% Memory map
% Priority Queue

% TODO :: priority queues

Within the RTX system, there are several pieces of information available
globally within all contexts and several that are exposed only within kernel
mode. Certain data structures, such as the message envelope, are used by
user processes. Other structures are used strictly by the kernel. Several
global variables are available for certain use cases. Constants are used to
define priority levels, IDs, and more. Stack usage is limited to standard
function uses, storing states before switching contexts, and for process
switching. Memory mapping is handled with a free list paired with an allocation
bit field for control.

\subsection{\textsc{Data Structures}}
The global data structures available have been split up into two categories:
user mode and kernel mode structures. This has been done to differentiate
between what is normally accessible in each context.

\subsubsection{\textsc{User Level}}
In user level mode there is one data structure available to the programmer
which is also available in kernel mode. This is the message envelope used to
send and receive messages between processes. It contains sender and receiver
process ids, the message type, pointers to the next and previous messages when
it is enqueued, values to control delay sending, and a data field defined at
the last 64 bytes of the envelope. The message type is defined in
Section~\ref{sec:global_constants}. To the user the only fields that are
modifiable in any meaningful way are the message type and data. All of the
other fields are modified by the kernel when the message is sent to another
process. If the user does not need to set the message type then this data
structure is not necessary as the only knowledge required for message sending
is that the data resides at an offset of 64 bits. In this case, the message
envelope is strictly used by the kernel.

\begin{table}[H]
    \caption{User Level Data Structures}
    \label{user_data_structs}
    \begin{tabular}{| l | p{8cm} |}
        \hline
        Structure & Definition \\
        \hline
        Message Envelope &
\begin{verbatim}
int sender_pid
int receiver_pid
enum message_type type
struct message_envelope* next
struct message_envelope* previous
int delay
int delay_start
unsigned char padding[36]
unsigned char data[64]
\end{verbatim} \\
        \hline
    \end{tabular}
\end{table}

\subsubsection{\textsc{Kernel Level}}


\subsection{\textsc{Constants}}
\label{sec:global_constants}


\subsection{\textsc{Memory}}
\label{sec:mem_management}
\section{\textsc{Primitives}}
% TODO :: words here

\subsection{\textsc{Release Processor}}
The release processor primitive is used to make a scheduling decision.
Currently, this happens when the currently running process calls
\texttt{release\_processor()}. After making a mode switch, the kernel makes a
decision on the next process to execute. When this decision is made, a context
switch is made to the selected process. See Listing~\ref{releaseproclisting}
for a high level pseudo code representation of this primitive

The kernel process decision currently uses several priority queues, four
priority levels with two sub-queues. Every priority level has a \texttt{READY} and a
\texttt{DONE} queue. Initially, all process in a priority level are in the
priority level's \texttt{READY} queue and after execution they are moved to the
\texttt{DONE} queue. In order to decide which process next, the kernel dequeues
the next process on the highest non-empty \texttt{READY} priority queue. If all
\texttt{READY} queues are empty, the \texttt{DONE} queues are moved to the
\texttt{READY} queues and the kernel tries to select a process again.

A context switch is made after the kernel selects the next process. If there is
a process running, the registers are saved on the stack and it's stack pointer
is moved to the PCB. It's state is marked as \texttt{STATE\_READY} to indicate
it's state has been saved and is capable of executing again. Next, the selected
process' stack pointer is restored. If the process is in the state
\texttt{STATE\_STOPPED}, meaning it has never been run before, the kernel
changes the state to \texttt{STATE\_RUNNING} and executes the process by
returning from the exception. On the other hand, if the process has been run
before, meaning that it is in the ready state, the data and address registers
are restored from the stack. In this case, the method returns normally.
Execution continues at the same it left off.

\lstset{caption={Pseudo code for \texttt{release\_processor()}},
        label=releaseproclisting}
\begin{lstlisting}

int release_processor() {
    mode switch to kernel mode using system call exception

    if all queues are empty:
        move DONE queues to READY queues

    dequeue next process from highest non-empty priority queue

    if there is a currently running process:
        save all of the registers on to the process stack
        save stack pointer to PCB
        set state to STATE_READY

    restore next process stack pointer
    if next process state is STOPPED:
        return from exception using the next process exception frame
    else if state is READY: 
        restore registers from stack
    
    mode switch to user mode using system call exception frame
    return RTX_SUCCESS
}
\end{lstlisting}

\subsection{\textsc{Set Process Priority}}
The \texttt{set\_process\_priority(...)} and \texttt{get\_process\_priority(...)}
primitives are used to set and get, respectively, the priorities of the 
specified process ID. 

Setting the process priority is shown as pseudo code in
Listing~\ref{setplisting}. First, the values of the process ID and the priority
are saved to registers as a mode switch to kernel mode needs to be performed.
These values are retrieved after the mode switch. After the mode switch, a
check is performed to ensure that the ID and priority is valid. If it isn't,
\texttt{RTX\_ERROR} is returned. Otherwise, another check is made to see if the
ID belongs to the currently running process and that there is a running
process. If this is true, the priority of the running process is simply changed 
and \texttt{RTX\_SUCCESS} is returned. If it is not the currently running 
process, the process is first removed from the appropriate queue based on its 
current priority level. Then the priority is updated to the new one and the 
process is enqueued onto the queue associated with the new priority with the 
same queue type as the one it was removed from. Finally, the function returns 
\texttt{RTX\_SUCCESS} to indicate successful completion.

\lstset{caption={Pseudo code for setting the process priority},
        label=setplisting}
\begin{lstlisting}
int set_process_priority(int pid, int priority) {
    move pid and priority into separate data registers
    mode switch to kernel mode
    
    retrieve pid and priority from the data registers

    if invalid PID or invalid priority:
        return RTX_ERROR
    
    if running process is the process to change:
        update running process to the new priority
    else:
        dequeue process from appropriate queue
        update priority of that process
        enqueue process to its new priority queue

    mode switch to user mode

    return RTX_SUCCESS
}
\end{lstlisting}

\subsection{\textsc{Get Process Priority}}
Compared to setting the priority level, getting a process' priority is straight
forward as can be seen in Listing~\ref{getplisting}. Similar to setting the
priority, the process ID is moved into a data register and system is switched 
into kernel mode. After retrieving the process ID, a check is made to ensure
that the process id is valid. If it is invalid \texttt{RTX\_ERROR} is returned.
The calling process should perform a check of its own to ensure that
the returned value is not a negative which indicates an error. Finally, the
process is looked up on the process table and the priority is returned from
that.

\lstset{caption={Pseudo code for getting the process priority},
        label=getplisting}
\begin{lstlisting}
int get_process_priority(int pid) {
    move pid into a data register
    mode switch to kernel mode

    retrieve pid from data register
    
    if the pid is invalid:
        return RTX_ERROR

    get process from the process table using pid
    
    mode switch to user mode

    return process priority
}
\end{lstlisting}

\subsection{\textsc{Request Memory Block}}
The \texttt{request\_memory\_block()} and \texttt{release\_memory\_block()}
primitives are used by the operating system to allocate and deallocate memory
blocks for processes.

Requesting a memory block is shown in pseudo code in
Listing~\ref{requestmemory}. First a mode switch to kernel mode is made. After
the mode switch, a check is made to see if any memory blocks are available. If
there are no available memory blocks, the process is moved to a blocked state
and the processor is released. If memory is available, the memory block is
popped off of the free list. The block index is retrieved and the appropriate memory
allocation bit is set high. Finally, the memory block address is returned.

\lstset{caption={Pseudo code for requesting memory blocks}, label=requestmemory}
\begin{lstlisting}
void* request_memory_block(){
    mode switch to kernel mode

    if no available memory blocks:
        running process state = blocked on memory
        release_processor()
    else:
        pop block of memory off the free list        
        get block index
        set corresponding bit in memory allocation field to 1

    return address of block
}
\end{lstlisting}

\subsection{\textsc{Release Memory Block}}
Releasing a memory block is shown in the pseudo code in
Listing~\ref{releasememory}. First a system call is made using the address of
the memory block as a parameter.  After the mode switch is performed, a check
is made to determine the index in the allocation field of the block. This index
is used to see if this block is currently allocated. If it is not then the
system returns \texttt{RTX\_ERROR}. If the block has been allocated then the
block is pushed back on to the free list and that bit in the memory allocation
field is set back to 0. Next, a check is made to see if there are any processes
currently blocked on memory. If any are found, the first blocked process has
its state set to ready and \texttt{RTX\_SUCCESS} is returned. If there are no
processes blocked on memory then \texttt{RTX\_SUCCESS} is returned right away.

\lstset{caption={Pseudo code for releasing memory blocks}, label=releasememory}
\begin{lstlisting}
int release_memory_block(void* memory_block){
    mode switch to kernel mode

    if memory block is not allocated:
        return RTX_ERROR
    else:
        push block back on to free list
        set correct bit in memory allocation field to 0

            if there are processes blocked on memory:
                set state of first blocked process to ready

        return RTX_SUCCESS        
}
\end{lstlisting}

\subsection{\textsc{Send Messages}}
The primitives \texttt{send\_message(...)} and \texttt{receive\_message(...)}
are used by processes to send and receive messages between them. Each process
has its own message queue which is setup upon process initialization.
When a process sends a message it is enqueued to the message queue of the
receiving process. When a process attempts to receive a message it polls its
own message queue.

Listing~\ref{sendmsg} describes the functionality of
\texttt{send\_message(...)}.  First, a system call is made with the
process id of the receiver and a pointer to the message envelope. The
system call then switches into kernel mode. Once the mode switch is
complete the message is then pushed on the end of the recipient's
message queue. There is no size limit on the message queue so this
will never cause an error. A check is then made to see if the
recipient process is currently blocked on receiving messages. If this
is the case the state of this process is set to ready. Then
\texttt{RTX\_SUCCESS} is returned.

\lstset{caption={Pseudo code for sending messages}, label=sendmsg}
\begin{lstlisting}
int send_message(int process_id, void* message_envelope){
    mode switch to kernel mode

    push message on to end of message queue

    if receiving process is blocked on messages:
        set state of receiving process to ready

    return RTX_SUCCESS
}
\end{lstlisting}

\subsection{\textsc{Receive Messages}}
Pseudo code for \texttt{receive\_message(...)} is shown in
Listing~\ref{receivemsg}. A system call is made to switch the system
into kernel mode. In this system call a pointer for storing the
process id of the sender is passed in as a parameter. Upon switching
to kernel mode the running process checks to see if its message queue
is empty. If the message queue is empty the process is now blocked on
messages. The processes' state is changed to reflect this and release
processor is called. If the queue is not empty the message is popped
off the queue, the process id of the sender is written to the
\texttt{sender\_id} pointer, and the message is returned.

\lstset{caption={Pseudo code for receiving messages}, label=receivemsg}
\begin{lstlisting}
void* receive_message(int* sender_id){
    mode switch to kernel mode

    if message queue for running process is empty:
        running process state = blocked on messages
        release_processor()
    else:
        pop message off queue
        *sender_id = process id of the message sender
        return message      
}
\end{lstlisting}

\subsection{\textsc{Delayed Send}}
The \texttt{delayed\_send(...)} primitive queues a message to be sent
after a specified delay. Messages are put onto a internal queue that
holds all delayed messages regardless of which process they belong to.
Messages in the queue are ordered by send time, with the soonest at
the front of the queue. This way, the check to see if a message should
be sent can be run in $O(1)$ time.

This function takes three parameters: the receiving process ID, a
pointer to a memory block containing the message, and the delay time
in milliseconds. After a mode switch to kernel mode, the parameters
are checked to make sure there is a valid receiver id and delay value;
invalid parameters return an \texttt{RTX\_ERROR}. If this check passes
then the message is inserted into the correct spot in the delayed
message queue.

Each time the timer i-process runs, it checks the head of the delayed
message queue to see if a message should be sent. If so, the message
is dequeued and forwarded to the correct process and the head is
checked again. The pseudo code for the delayed send is in
Listing~\ref{delay}. The pseudo code and implementation information
for the timer forwarding is shown below in the timer i-process
section.

\lstset{caption={Pseudo code for delayed send },label=delay}
\begin{lstlisting}
int delayed_send(int process_id, void* message_envelope, int delay):
    switch to kernel mode

    if (receiver_id is invalid || delay is invalid):
        return RTX_ERROR
    else:
        set message sender id
        set message receiver id
        set delay information
        insert into delay queue

        return RTX_SUCCESS
\end{lstlisting}



\section{\textsc{Processes}}
\subsection{\textsc{Null Process}}
The null process was implemented to adhere to the project specifications. It
has a priority of four and a process ID of zero. The process method is defined 
as an infinite loop that constantly calls \texttt{release\_processor()}. It is 
added to the process table through a call to \texttt{init\_processes(...)}. As
this process is required to always be at process ID zero, care should be taken
to ensure that this is not overwritten by other processes in the initialization
methods. This process is outlined in Listing~\ref{nullplisting}.

\lstset{caption={Pseudo code for the null process},
        label=nullplisting}
\begin{lstlisting}
void process_null() {
    while (true):
        release_processor();
}
\end{lstlisting}

\subsection{\textsc{KCD Process}}
The \texttt{process\_kcd()} blocks until a message is received. Once it
receives a message, the message type is checked to see if it is either
key input or command registration. If it is neither, the message is
released. 

When the message type is key input, the message is checked to see if
it matches to a previously registered command. If a match is found the
message is forwarded to the registered process for that command.

When the message type is a command registration, the PID of the
registering process is saved along with the command itself. The
command character is passed in the message body. The pseudo code for
the KCD can been seen in Listing~\ref{kcd}.

\lstset{caption={Pseudo code for implementing the KCD Processes}, label=kcd}
\begin{lstlisting}
void process_kcd():
    while(1):
        receive a message
        if (message type = key input):
            if (message body = registered command):
                send message to registered process
        else if (message type = command registration):
            save the PID and command to be registered
        else:
            release message
\end{lstlisting}

\subsection{\textsc{CRT Process}}
The \texttt{process\_crt\_display()} blocks until a message is
received. If type is correct, then the message data is sent to the
UART and the message is released. Pseudo code for the
process can be found in Listing~\ref{crt}.

\lstset{caption={Pseudo code for the CRT process}, label=crt}
\begin{lstlisting}
void process_crt_display():
    while(1):
        receive a message
        if (message type = output || message type = key input):
            send message data to uart
        release message
    }
}
\end{lstlisting}

\subsection{\textsc{UART I-Process}}
The UART i-process is run in response nice to both transmit ready and
receive ready UART interrupts. After either of these interrupts fires,
a context switch is immediately made to the UART i-process. Once it
runs, the UART status register is checked to determine whether a read
or a write is to occur.

When reading in data, all characters are appended to a buffer until a carriage
return is read in. At this point, a null terminating character is appended.  A
check is made to determine the first value in the string buffer. If it is a
percent (\texttt{\%}), a message is created with the string buffer copied into its data
field. This message is sent to the KCD. If it is an exclamation mark (!), the
string buffer is sent to the UART debug decoder. After this, the character read
in is echoed out again through a transmit interrupt request and interrupts are
re-enabled. 

If it is a write, a character is written to serial port and transmit
ready interrupt is masked. If a carriage return was written, then a
newline character is also queued to be written. A special case is
considered for when the message type is specified to not require any
newlines. Pseudo code for this process is below as Listing
~\ref{uart}.

\lstset{caption={Pseudo code for the UART I-Process}, label=uart}
\begin{lstlisting}
void i_process_uart():
    while(1):
        determine uart state

        if(state = read):
            read in character
            append character to string buffer
            if (character = carriage return):
                append null to string buffer            

                if (buffer[0] = '%'):
                    create a new message
                    copy string buffer into message data field
                    send message to KCD
                else if (buffer[0] = '!'):
                    send string buffer to uart debug decoder

            echo the in character by triggering a transmit interrupt

            enable interrupts
        else if (state = write):
            write character to serial port
            mask transmit ready interrupt
            if (character = carriage return AND requires a new line):
                print newline
            
        release processor
\end{lstlisting}

\subsection{\textsc{Timer I-Process}}
The \texttt{i\_process\_timer()} process runs immediately in response
to timer interrupts. Upon receiving an interrupt, this process
performs two tasks. First, it increments its internal time
counter. Next, the delayed messages queue is checked to see if there
are any messages to send out. All messages that were scheduled to be
sent at the current time or sooner are sent. At this point, this
process will relinquish control to the processor. See
Listing~\ref{timer} for the pseudo code for this process.

\lstset{caption={Pseudo code for the UART I-Process}, label=timer}
\begin{lstlisting}
void i_process_timer():
    timer = 0
    while(1){
        release processor
        increment timer

        if(delayed messages queue is not empty):
            while (queue head send time <= current time):
                remove message from delayed messages queue
                send message to original receiver
\end{lstlisting}

\subsection{\textsc{Wall Clock Process}}
The \texttt{process\_wall\_clock()} is used to display the current OS
time on UART1. This clock is given an initial time by sending the
\texttt{\%WS} command and is updated every second. Clock time is printed
until the \texttt{\%WT}. Internally, the current time is stored in seconds.

The wall clock receives time updates by repeatedly sending delayed
messages to itself. Every time it receives one of these messages, it
updates its internal clock, prints out the time, and sends itself
another delayed message. The delayed messages are sent with a one
second delay. Once the internal clock reaches 86400 seconds (24 hours)
it is reset to 0. These messages are only sent if the wall clock has
been started with the \texttt{\%WS} command.

When the wall clock receives a message from the KCD process it parses
the command and performs one of two actions. If the command was
\texttt{\%WS}, it parses the time from the command, sets the internal
clock, and sends itself a delayed message. The user input is
validated, and if it does not pass the clock is not changed.

Pseudo code for the wall clock process can be seen at
Listing~\ref{wallclock}.

\lstset{caption={Pseudo code for the wall clock process}, label=wallclock}
\begin{lstlisting}
void process_wall_clock():
    while (1):
        receive message
        if (sender_id = wall clock pid):
            increment clock

            if (clock = 86400):
                reset clock

            if (clock is currently on):
                send a delayed message to itself with a delay of one second
                convert clock to hours, minutes and seconds
                convert hours, minutes, seconds to time string
                send time string as message to CRT

        else if (sender_id = KCD pid):
            if (%WS):
                if (clock is currently off):
                    send a delayed message to itself with delay of one second

                parse string and validate input
                store hours, minutes and seconds
            else if (%WT):
                turn off clock
        release message
\end{lstlisting}

\subsection{\textsc{Set Priority Command Process}}
The \texttt{process\_set\_priority\_command()} is a process that
allows the user to change the priority level of any process through a
command. This process accepts data in the form \texttt{\%C process\_id 
priority\_level}. When the process is initialized it registers the
\texttt{\%C} command. It then waits to receive a message.  Upon receiving a
message it skips the \texttt{\%C} characters and begins parsing the
data. While parsing, the process id of the process whose priority is
to be changed and the new priority level are extracted. Checks are
made to ensure that proper process ids and priority values were
provided and that nothing else was entered after the priority value. A
kernel function call of \texttt{set\_process\_priority(...)} is then
made using the parsed process id and priority value. This function
call is responsible for the actual changing of the priority level. The
memory block and the processor are then released.  Pseudo code for
this process can be seen at Listing \ref{lst_set_prio_cmd_proc}.

\lstset{caption={Pseudo code for the set priority command process}, 
        label=lst_set_prio_cmd_proc}

\begin{lstlisting}
void process_set_priority_command():    
    register %C Command

    while (1):
        receive a message

        skip %C in message

        parse process id
        validate process id

        parse priority level
        validate priority level

        validate line endings
        
        call set_process_priority()
            update the priority level of the specified process

        release memory block;
        release processor;
\end{lstlisting}

\subsection{\textsc{Test Processes A, B, C}}
The system also has three test processes implemented. These processes are used
to stress test our system by depleting memory blocks. \texttt{Process A}
requests a memory block and on receiving one sends a message to \texttt{Process
B}. \texttt{Process B} receives the message and then sends the message to
\texttt{Process C}. \texttt{Process C} then gets a message and checks if the
number of messages it has received is divisible by 20. If it is it prints out
"Process C" to the UART1. Then the process hibernates for 10 seconds. All of
these processes loop forever inorder to deplete memory. In order to have these
processes work properly within our system \texttt{Process C} was assigned a
priority of 1 and the other two processes were assigned a priority of 2.

\section{\textsc{Software Interrupt Handler}}

% TODO: ADD MORE?

The system uses a single software interrupt handler for performing context
switches into kernel mode for kernel functions. It is initialized in the 0th
location on the trap table. The trap is only called from the function
\texttt{do\_system\_call(int call\_id, int* args, int num\_args)} which is used
to abstract common code used by all the primitives. The software handler itself
that is called by the trap mnemonic is \texttt{system\_call()} which retrieves
the parameters from data registers 1 to 4. It calls the appropriate kernel
level function based on the call id parameter. The call ids are defined in an
enumerator. 

The software interrupt handler, as described in Listing~\ref{lst_system_call},
is installed in vector 0 of the trap table. The operation of this interrupt
handler retrieves the parameters from the data registers, saves the rest of
the user process state into the stack, and calls the appropriate kernel level
function. The parameters retrieved are passed in as arguments for the function.
The function return value is stored in one of the data registers to be passed
back to the function that called the trap. Finally, the function returns from
exception to switch back from kernel mode into user mode.

\lstset{caption={Software Interrupt Handler description},
        label=lst_system_call}
\begin{lstlisting}
void system_call() {
    disable interrupts

    retrieve call id and arguments from data registers

    save current user process state

    use call id to select the appropriate kernel level function 
    pass in the arguments as parameters to the function

    restore user process state

    load return value into a data register

    return from exception
}
\end{lstlisting}

\section{\textsc{Hardware Interrupt Handlers}}
% data structures, functional outline, pseudocode
Two hardware interrupt handlers were implemented. The first is for the timer
interrupts and the second is for the UART for keyboard input from the user.
They both use very similar logic. All of the data and address registers are
moved onto the stack after interrupts are disabled. Then the processor is
preempted with the the respective interrupt process selected as the next
running process. In the case of the timer interrupt, the interrupt event is
acknowledge to prevent the interrupt from firing multiple times for the same
event. After the interrupt process releases the processor, and the processor
returns back to the interrupted process, the data and address registers are
restored from the stack and the interrupt handler returns from the exception.
See Listing~\ref{listing_hwi_handler} for the pseudo code description of the
two interrupt handlers.

\lstset{caption={Hardware Interrupt Handler Pseudo Code},
        label=listing_hwi_handler}
\begin{lstlisting}
void interrupt_handler() {
    disable interrupts
    move data and address registers onto the stack

    acknowledge the interrupt
    preempt processor with the appropriate interrupt process

    restore data and address registers from the stack

    return from exception
}
\end{lstlisting}

Interrupt service routines are loaded into the interrupt vector table and
initialized with the required settings in the initialization period of the
system startup. This is described in Section~\ref{sec:rtx_init}.

\section{\textsc{Hot Keys}}
% What they are, their functionality, pseudocode 

% TODO: Edit this for tense
A total of five hot key commands were implemented to assist in debugging and
testing of the system and user processes. A hot key decoder function was used to
interpret the inputted command from the UART interrupt process. The interrupt
process checks the first character for an exclamation mark ('!') and calls the
hot key function directly with the input string as a parameter. See
Table~\ref{hot_keys_table} for a summary of the hot key strings and
descriptions. All of the code related to the debug hot keys are wrapped in
\texttt{\#ifdef \_DEBUG\_HOTKEYS} and \texttt{\#endif} to ensure that in normal
operation of the system that this code will not interfere.

\begin{table}[h]
    \caption{Hot Keys and Descriptions}
    \label{hot_keys_table}
    \begin{tabular}{| r | r |}
        \hline
        Hot Key Command & Description \\
        \hline
        !RQ & Print ready processes and priorities \\ 
        !BMQ & Print processes blocked on memory \\ 
        !BRQ & Print processes blocked on receiving messages \\ 
        !FM & Print the current number of free memory blocks \\ 
        !M & Print the last $i$ messages sent where $i$ is the debug message
        log size \\
        \hline
    \end{tabular}
\end{table}

Each of the debugging functions are kernel level calls as they require access
to data structures in the kernel. This is done using the same soft interrupt
method as the the rest of the kernel routines. The hot key command decoder
uses a similar parsing method as the \texttt{process\_wall\_clock()} process to
interrupt which kernel routine to run. The debugging processes related to the
ready queues and blocked queues use the kernel print queue function as
described in Listing~\ref{lsting_k_print_queue}. The specific queue debugging 
function simply passes in the relavent queue into the queue print function.

\lstset{caption={Kernel Print Queue Psuedo Code}, 
        label=lsting_k_print_queue}
\begin{lstlisting}
int queue_debug_print(process_queue queue[]) {
    for each process in the queue:
        print process ID
        print process priority
}
\end{lstlisting}

The hot key, \texttt{!FM}, simply outputs the number of free blocks and the
allocation bit field. The allocation bit field is described in
Section~\ref{sec:mem_management}.

The last hot key, \texttt{!M}, is the most complicated of the five. In each of
the kernel functions related to message sending and receiving, a copy of each
message is stored on a circular buffer that contains a data structure that
maintains a record of the sender ID, receiver ID, message type, the first 16
bytes of data, and the time stamp of the message. Seperate circular buffers are
used for the sent messages and received messages to assist in debugging if a
message has been sent but not received.  When the hot key is called, it goes
through every message in both queues and outputs first all the sent messages
and then all of the received messages. The number of messages store in the two
queues can be changed by modifying the \texttt{DEBUG\_MESSAGE\_LOG\_SIZE}
definition. 

\section{\textsc{Initialization}}
\label{sec:rtx_init}
The RTX initialization happens in four phases: process, priority queues, 
interrupts, and the memory management system. Pseudo code for everything except 
memory management can be seen in Listing~\ref{initlisting} while the pseudo code 
for memory management is located in Listing~\ref{meminit}.

The process initialization occurs in \texttt{init\_processes()} and is done in 
two stages. The first is an initialization of the test processes and the second 
is the initialization of the null and user processes. For each process, a PCB is 
created on the process table. This PCB is then added to the kernel's own process 
list but not yet added to a priority queue. Then, for each process in the 
kernel's process list, a stack pointer is setup and an exception frame is added 
to the process' new stack. The format/vector word portion of the exception frame 
is set to 0x4000 to represent a 4 byte aligned stack and the SR portion is 
0x0000. The program counter in this frame is set to the entry point of 
the process.

Next, priority queues are initialized. The data structures are already
allocated but are not initialized. To ensure proper operation, the queues are
put into a consistent empty state. After the queues are setup, the processes
are added to appropriate queue.

Then, the interrupts are setup. First, the vector base register (VBR) for the
interrupts is set to memory location 0x10000000. Next, the soft interrupt for
\texttt{system\_call()} is installed to the VBR at vector 0.

\lstset{caption={Pseudo code for RTX initialization},label=initlisting}
\begin{lstlisting}
void init() {
    init_processes()
    init_priority_queues()
    init_interrupts()
}

void init_processes() {
    init_test_processes()
    init_user_processes()
    for each process in process list:
        initialize stack pointer
        push PC portion of exception frame onto the stack
        push F/V and SR portions of exception frame onto the stack
}

void init_test_processes() {
    for each test process: 
        setup PCB
        add to process list
}

void init_user_processes() {
    user process list:
        null process
        Process A
        Process B
        Process C
        Uart
        Timer
        CRT
        KCD
        Wall Clock Display
        Set Priority Process

    for each process in user process list:
        setup PCB
        add to process list
}

void init_priority_queues() {
    for each process: 
        add process to correct priority queue
}

void init_interrupts() {
    initialize VBR to 0x10000000
    install system call ISR at vector 0
}
\end{lstlisting}

The final phase of the RTX initialization is the memory management phase.
Memory blocks are allocated beginning at the start of free memory. The function
call \texttt{init\_memory(void* memory\_start)}, as seen in
Listing~\ref{meminit}. A free list is used to implement linking and management
of the memory blocks. Over-allocation and double deallocation prevention is 
handled by a bit field containing a bit for each memory block. When the memory 
is allocation the associated bit is set to 1 and for when it is released it is 
set back to 0.

\lstset{caption={Pseudo code for RTX initialization of memory
        management},label=meminit}
\begin{lstlisting}
void init_memory(void* memory_start){
    
    head of memory = start of free memory + (number of memory blocks - 1)*(size of memory block)
    
    for each block in free list:
        store address of next free memory block in current block

    initialize the memory allocation bit field to 0
}
\end{lstlisting}


\section{\textsc{Implementation}}

\chapter{\textsc{Major Design Changes}}

\chapter{\textsc{Measurements}}

\chapter{\textsc{Leasons Learned}}
\end{document}
