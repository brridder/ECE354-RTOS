\documentclass[oneside]{report}
\usepackage{listings}
\usepackage[hmargin=1.4in,vmargin=1.0in]{geometry}
\usepackage[T1]{fontenc}
\usepackage[latin1]{inputenc}

\begin{document}
\lstset{language=C, 
        frame=single, 
        breaklines=true,
        basicstyle=\small\ttfamily,
        columns=fullflexible}

% Pre-amble - title

\title{ECE354: RTX Project Final Report}
\author{Ben Ridder - brridder \\
Casey Banner - cccbanne \\
David Janssen - dajjanss }
\date{\today}

\maketitle

\tableofcontents

\chapter{\textsc{Software Design}}

\section{\textsc{Introduction}}

\section{\textsc{Global Information}}

\section{\textsc{Primitives}}

\section{\textsc{Processes}}

\section{\textsc{Software Interrupt Handlers}}

\section{\textsc{Hardware Interrupt Handlers}}

\section{\textsc{Hot Keys}}
% What they are, their functionality, pseudocode
A total of five hot key commands were implemented to assist in debugging and
testing of the system and user processes. A hot key function was used to
interpret the inputted command from the UART interrupt process. The interrupt
process checks the first character for an exclamation mark ('!') and calls the
hot key function directly with the intput string as a parameter. See
Table~\ref{hot_keys_table} for a summary of the hot key strings and descriptions.

\begin{table}[h]
    \caption{Hot Keys and Descriptions}
    \label{hot_keys_table}
    \begin{tabular}{| r | r |}
        \hline
        Hot Key Command & Description \\
        \hline
        !RQ & Print ready processes and priorities \\ 
        !BMQ & Print processes blocked on memory \\ 
        !BRQ & Print processes blocked on receiving messages \\ 
        !FM & Print the current number of free memory blocks \\ 
        !M & Print the last $i$ messages sent where $i$ is the debug message
        log size \\
        \hline
    \end{tabular}
\end{table}

\section{\textsc{Initialization}}

\section{\textsc{Implementation}}

\chapter{\textsc{Major Design Changes}}

\chapter{\textsc{Measurements}}

\chapter{\textsc{Leasons Learned}}
\end{document}
