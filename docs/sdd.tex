\documentclass[oneside]{report}
\usepackage{listings}
\usepackage[hmargin=1.4in,vmargin=1.5in]{geometry}
\usepackage[T1]{fontenc}
\usepackage[latin1]{inputenc}
\usepackage{float}
\begin{document}
\lstset{language=C, 
        frame=single, 
        breaklines=true,
        basicstyle=\small\ttfamily,
        columns=fullflexible}

% Pre-amble - title

\title{ECE354: RTX Project Final Report}
\author{Ben Ridder - brridder \\
Casey Banner - cccbanne \\
David Janssen - dajjanss }
\date{\today}

\maketitle

\tableofcontents
\listoftables
\lstlistoflistings

\chapter{\textsc{Software Design}}
The following chapter is concerned with the specific software design of the
RTX operating system. The first section, Section~\ref{sec:overview}, covers an
overview of the system. Next, in Section~\ref{sec:global_info}, all of the data
structures, global variables, constants, and more is covered as background for
the primitives and processes. Each kernel primitive is discussed with pseudo
code within Section~\ref{sec:primitives}. Processes, both system and user, are
covered next in Section~\ref{sec:processes}. Next, software and hardware
interrupt handlers are covered in Section~\ref{sec:soft_isr} and
Section~\ref{sec:hard_isr} respectively. Section~\ref{sec:hot_keys} deals with
the implementation of the system hot keys and the functionality of each hot
key. Finally, after initialization in Section~\ref{sec:rtx_init}, the implementation
details of the entire system is explained in Section~\ref{sec:implementation}.

\section{\textsc{Overview}}
\label{sec:overview}

At a high level, execution of the operating system consists of two
main parts: initialization, and running processes. When execution
begins, the operating system initializes memory, processes, priority
queues, and interrupts. Once the initialization is complete, the
operating system is in a state where it can run processes and respond
to system calls as well as hardware interrupts. System call services
are provided to processes running within the operating system in the
form of a software interrupt handler. After initialization, a process
is scheduled and then run. Whenever a system call occurs or a hardware
interrupt is handled, the operating system may decide to schedule a
new process. The overall flow diagram of the operating system can be
found in Figure~\ref{flowchart}.

\begin{figure}[H]
    \label{flowchart}
    \center {
      \caption{Operating system flow diagram}
      \scriptsize
      \setlength{\unitlength}{2.0em}
      \begin{picture}(18.500000,30.500000)(-6.500000,-30.500000)
% picture environment flowchart generated by flow 0.99f
\put(2.5000,-0.5000){\oval(5.0000,1.0000)}
\put(0.0000,-1.0000){\makebox(5.0000,1.0000)[c]{\shortstack[c]{
Main entry point
}}}
\put(2.5000,-1.0000){\vector(0,-1){1.0000}}
\put(0.0000,-5.0000){\framebox(5.0000,3.0000)[c]{\shortstack[c]{
Init Memory\\
Init Processes\\
Init Priority queues\\
Init Interrupts
}}}
\put(2.5000,-5.0000){\vector(0,-1){1.0000}}
\put(0.5000,-7.0000){\framebox(4.0000,1.0000)[c]{\shortstack[c]{
Release Processor
}}}
\put(2.5000,-7.0000){\vector(0,-1){1.0000}}
\put(0.5000,-9.0000){\framebox(4.0000,1.0000)[c]{\shortstack[c]{
Schedule Process
}}}
\put(2.5000,-9.0000){\vector(0,-1){1.0000}}
\put(0.5000,-11.0000){\framebox(4.0000,1.0000)[c]{\shortstack[c]{
Run Process
}}}
\put(2.5000,-11.0000){\line(0,-1){1.0000}}
\put(2.5000,-12.0000){\line(-1,0){1.0000}}
\put(1.5000,-12.0000){\vector(0,1){1.0000}}
\put(0.5000,-10.5000){\vector(-1,0){1.0000}}
\put(-2.5000,-10.5000){\oval(4.0000,2.0000)}
\put(-4.5000,-11.5000){\makebox(4.0000,2.0000)[c]{\shortstack[c]{
Hardware\\
interrupt\\
handler
}}}
\put(-2.5000,-11.5000){\vector(0,-1){1.0000}}
\put(-5.5000,-13.5000){\makebox(6.0000,1.0000)[c]{\shortstack[c]{
Run relevant I-Process
}}}
\put(-5.3333,-12.5000){\line(1,0){6.0000}}
\put(-5.6667,-13.5000){\line(1,0){6.0000}}
\put(-5.6667,-13.5000){\line(1,3){0.3333}}
\put(0.3333,-13.5000){\line(1,3){0.3333}}
\put(-2.5000,-13.5000){\vector(0,-1){1.0000}}
\put(-2.5000,-15.5000){\oval(5.0000,2.0000)}
\put(-5.0000,-16.5000){\makebox(5.0000,2.0000)[c]{\shortstack[c]{
Return from\\
hardware interrupt\\
handler
}}}
\put(-2.5000,-16.5000){\line(0,-1){1.0000}}
\put(-2.5000,-17.5000){\line(-1,0){4.0000}}
\put(-6.5000,-17.5000){\line(0,1){9.0000}}
\put(-6.5000,-8.5000){\vector(1,0){7.0000}}
\put(4.5000,-10.5000){\vector(1,0){1.0000}}
\put(8.0000,-10.5000){\oval(5.0000,2.0000)}
\put(5.5000,-11.5000){\makebox(5.0000,2.0000)[c]{\shortstack[c]{
System call\\
interrupt\\
handler
}}}
\put(8.0000,-11.5000){\vector(0,-1){1.0000}}
\put(5.0000,-13.5000){\framebox(6.0000,1.0000)[c]{\shortstack[c]{
Switch to kernel mode
}}}
\put(8.0000,-13.5000){\vector(0,-1){1.0000}}
\put(5.5000,-15.5000){\makebox(5.0000,1.0000)[c]{\shortstack[c]{
Handle system call
}}}
\put(5.6667,-14.5000){\line(1,0){5.0000}}
\put(5.3333,-15.5000){\line(1,0){5.0000}}
\put(5.3333,-15.5000){\line(1,3){0.3333}}
\put(10.3333,-15.5000){\line(1,3){0.3333}}
\put(8.0000,-15.5000){\vector(0,-1){1.0000}}
\put(6.0000,-18.5000){\line(1,1){2.0000}}
\put(6.0000,-18.5000){\line(1,-1){2.0000}}
\put(10.0000,-18.5000){\line(-1,-1){2.0000}}
\put(10.0000,-18.5000){\line(-1,1){2.0000}}
\put(6.0000,-20.5000){\makebox(4.0000,4.0000)[c]{\shortstack[c]{
Is running\\
process\\
now\\
blocked?
}}}
\put(6.0000,-17.9000){\makebox(0,0)[rt]{N}}
\put(10.0000,-17.9000){\makebox(0,0)[lt]{Y}}
\put(10.0000,-18.5000){\line(1,0){2.0000}}
\put(12.0000,-18.5000){\line(0,1){10.0000}}
\put(12.0000,-8.5000){\vector(-1,0){7.5000}}
\put(6.0000,-18.5000){\line(-1,0){2.0000}}
\put(4.0000,-18.5000){\vector(0,1){7.5000}}
\put(8.0000,-20.5000){\vector(0,-1){1.0000}}
\put(6.0000,-23.5000){\line(1,1){2.0000}}
\put(6.0000,-23.5000){\line(1,-1){2.0000}}
\put(10.0000,-23.5000){\line(-1,-1){2.0000}}
\put(10.0000,-23.5000){\line(-1,1){2.0000}}
\put(6.0000,-25.5000){\makebox(4.0000,4.0000)[c]{\shortstack[c]{
Is running\\
process\\
preempted?
}}}
\put(6.0000,-22.9000){\makebox(0,0)[rt]{N}}
\put(10.0000,-22.9000){\makebox(0,0)[lt]{Y}}
\put(10.0000,-23.5000){\line(1,0){2.0000}}
\put(12.0000,-23.5000){\line(0,1){5.0000}}
\put(6.0000,-23.5000){\line(-1,0){2.0000}}
\put(4.0000,-23.5000){\line(0,1){5.0000}}
\put(8.0000,-25.5000){\vector(0,-1){1.0000}}
\put(6.0000,-28.5000){\line(1,1){2.0000}}
\put(6.0000,-28.5000){\line(1,-1){2.0000}}
\put(10.0000,-28.5000){\line(-1,-1){2.0000}}
\put(10.0000,-28.5000){\line(-1,1){2.0000}}
\put(6.0000,-30.5000){\makebox(4.0000,4.0000)[c]{\shortstack[c]{
Was\\
processor\\
released?
}}}
\put(6.0000,-27.9000){\makebox(0,0)[rt]{N}}
\put(10.0000,-27.9000){\makebox(0,0)[lt]{Y}}
\put(10.0000,-28.5000){\line(1,0){2.0000}}
\put(12.0000,-28.5000){\line(0,1){5.0000}}
\put(6.0000,-28.5000){\line(-1,0){2.0000}}
\put(4.0000,-28.5000){\line(0,1){5.0000}}
\end{picture}

      \normalsize
    }
\end{figure}

\pagebreak

\section{\textsc{Global Information}}
\label{sec:global_info}
% Data structures (queues, PCB, message formats, tracebuffer, etc)
% Variables
% Constants (PIDS, msgID, return codes, status codes, msg types, etc)
% Stacks
% Memory map

Within the RTX system, there are several pieces of information available
globally within all contexts and several that are exposed only within kernel
mode. Certain data structures, such as the message envelope, are used by user
processes whereas other structures are used strictly by the kernel. Several
global variables are available for certain use cases. Constants are used to
define priority levels, IDs, and more. Stack usage is limited to standard
function uses, storing states before switching contexts, and for process
switching. Memory mapping is handled with a free list paired with an allocation
bit field for control.

\subsection{\textsc{Data Structures}}
The global data structures available have been split up into two categories:
user mode and kernel mode structures. This has been done to differentiate
between what is normally accessible in each context.

\subsubsection{\textsc{User Level}}
In user level mode there is one data structure available to the programmer,
which is also available in kernel mode. This is the message envelope used to
send and receive messages between processes. It contains sender and receiver
process ids, the message type, pointers to the next and previous messages when
it is enqueued, values used to control delay sending, and a data field defined
at the last 64 bytes of the envelope. The message type is defined in
Section~\ref{sec:global_constants}. To the user the only fields that are
modifiable in any meaningful way are the message type and the message data. All
of the other fields are modified by the kernel when the message is sent to
another process. If the user does not need to set the message type then this
data structure is not necessary as the only knowledge required for message
sending is that the data resides at an offset of 64 bits. In this case, the
message envelope is used strictly by the kernel. The message envelope
definition can be seen in Table~\ref{user_data_structs}.

\begin{table}[H]
    \caption{User Level Data Structure}
    \label{user_data_structs}
    \center{
    \begin{tabular}{| l | p{8cm} |}
        \hline
        Structure & Definition \\
        \hline
        Message Envelope &
\begin{verbatim}
int sender_pid
int receiver_pid
enum message_type type
struct message_envelope* next
struct message_envelope* previous
int delay
int delay_start
unsigned char padding[36]
unsigned char data[64]
\end{verbatim} \\
        \hline
    \end{tabular}
    } % Center
\end{table}

\subsubsection{\textsc{Kernel Level}}
The system includes several kernel level data structures such as: the process
control block, two queues, and a debug message envelope. The process control
block contains all the information required by the system to manage and
schedule the process. Both queues are implemented with a doubly linked list to
ensure $O(1)$ operation for both popping off the head and pushing to the tail.
Message queue is used for storing message envelopes associated with a specific
process. The process queue is used for the priority queues. The structures are
defined in Table~\ref{kernel_data_structs}

\begin{table}[H]
    \caption{Kernel Level Data Structures}
    \label{kernel_data_structs}
    \center{
    \begin{tabular}{| l | p{8cm} |}
    \hline
    Structure & Definition \\
    \hline
    Process Control Block &
\begin{verbatim}
int pid
int priority
void* stack
int stack_size
void (*entry)()
bool is_i_process
enum process_state state
enum queue_type queue
message_queue messages
struct process_control_block* next
struct process_control_block* previous
\end{verbatim} \\

    \hline
    Message Queue & 
\begin{verbatim}
message_envelope* head
message_envelope* tail
\end{verbatim} \\
    \hline

    Process Queue &
\begin{verbatim}
process_control_block* head 
process_control_block* tail 
\end{verbatim} \\
    \hline

    Debug Message Envelope &
\begin{verbatim}
int sender_pid
int receiver_pid
enum message_type type
unsigned char data[16]
int time_stamp
\end{verbatim} \\
    \hline

    \end{tabular}
    } % Center
\end{table}

% Stacks
% Memory map
% Priority Queue

\subsection{\textsc{Variables}}
Within the kernel, there are several variables that are used for various
components. For memory, there is a pointer to the head of the free list, a bit
field for determining memory allocations, and a pointer to the end of free
memory. Several process queues are used for containing the ready, blocked on
message, and blocked on memory processes in FIFO order. These are stored in an
array of process queues in order to generalize queue operations. If debugging 
hot keys are enabled, then two circular buffers for the ten most recently sent 
and received messages are available in the kernel. Finally, a counter used for
timing is incremented by the timer ISR and read by various kernel functions.

\subsection{\textsc{Constants}}
\label{sec:global_constants}

% Constants (PIDS, msgID, return codes, status codes, msg types, etc)

All of the processes have a predefined process ID as outlined in
Table~\ref{table_proc_ids}. There is a total of 15 unique processes and there
is no capability to add extra processes during system execution. 

\begin{table}[H]
    \caption{Process IDs}    
    \label{table_proc_ids}
    \center{
    \begin{tabular}{| r | c || r | c |}
    \hline
    Process & Process ID & Process & Process ID \\
    \hline
    \hline 
    \texttt{Null Process} & 0 & \texttt{User Process B} & 8 \\
    \texttt{Test Process 1} & 1 & \texttt{User Process C} & 9 \\
    \texttt{Test Process 2} & 2 & \texttt{UART Interrupt Process} & 10 \\
    \texttt{Test Process 3} & 3 & \texttt{Timer Interrupt Process} & 11 \\ 
    \texttt{Test Process 4} & 4 & \texttt{CRT Display Process} & 12 \\
    \texttt{Test Process 5} & 5 & \texttt{KCD Process} & 13 \\
    \texttt{Test Process 6} & 6 & \texttt{Wall Clock Process} & 14 \\
    \texttt{User Process A} & 7 & \texttt{Set Priority Process} & 15 \\
    \hline
    \end{tabular}
    } % Center
\end{table}

There are six message types defined using an enumerator available to the
processes for specific duties. See Table~\ref{table_msg_types} for the type
name, process, and description of each type.

\begin{table}[H]
\small
    \caption{Message Types}
    \label{table_msg_types}
    \center{
    \begin{tabular}{| r | r | p{5cm} |}
    \hline
    Message Type & Receiving Process & Description \\
    \hline
    \hline
    \texttt{MESSAGE\_CMD\_REG} & KCD Process & Used to register the command
    sequence in the data field. \\
    \texttt{MESSAGE\_KEY\_INPUT} & KCD Process & The data field contains a
    command string that is interpreted and forwarded to the registered
    process. \\
    \texttt{MESSAGE\_OUTPUT} & CRT Display Process & The data field contains
    a string to output to the display. \\ 
    \texttt{MESSAGE\_OUTPUT\_NO\_NEWLINE} & CRT Display Process & The data
    field contains a string to output to the display but does not include a
    newline. Used by the wall clock process. \\
    \texttt{MESSAGE\_COUNT\_REPORT} & User Process C & The data field contains
    the current count from User Process A. \\
    \texttt{MESSAGE\_WAKE\_UP\_10} & User Process C & Sent on a delay of 10
    seconds to alert User Process C to wake up and continue. \\
    \hline
    \end{tabular}
    } % Center

\end{table}


Two return codes are used by functions in the system. The first,
\texttt{RTX\_SUCCESS}, is defined as $0$ and is used for indicating a success 
within the system. The other, \texttt{RTX\_ERROR}, is defined as $-1$ and is 
used for indicating a failure. This maintains consistency between different 
functions and makes the code more readable for checking return conditions.

\subsection{\textsc{Stacks}}

The only stacks used in the system are the process stacks. These are used for
saving and restoring the process state when making a mode switch and for
switching processes. The stacks are also used for allocation of local variables
and function calls as determined by the compiler.

\subsection{\textsc{Memory}}
\label{sec:mem_management}
The memory system consists of two components that are managed by the request
and release memory block primitives. The first is a free list containing a
total of 32 128-byte memory blocks. Each memory block is linked together using
the first word as the pointer to the next block of free memory. When a block is
requested, the head is sent to the requester and the new head becomes the
memory location that the old one pointed to. Putting a block back onto the free
list requires setting the first word of the block to the pointer of the current
head and the new head is now that block. The benefit to this approach is that
the blocks can be returned onto the free list in any order without affecting
the operation of this data structure. All free list operations are done in
$O(1)$ time.

The other component is a bit field. Since only 32 blocks of memory are
allocated by the system, a bit field of length 32 bits is used to represent
each block one to one. If the value of the bit for the corresponding block is a
$1$, then this indicates that the block has been already been requested by a
process. When the bit is $0$, then the block is somewhere on the free list and
can be freely allocated to a requester. The enables over-allocation and
double-deallocation checks to be done in $O(1)$ time. The bit is set high (1)
when allocated and set low (0) when released.

\subsection{\textsc{Priority Queues}}
Priority queues are done using a standard queue data structure. The only
difference is that there is a queue for each priority level. Hence, for the
ready and blocked queues an array of queues is used with the index of the queue
as the priority level. As described earlier, every queue is implemented using a
doubly linked list so that nearly every operation occurs in $O(1)$ time.

\pagebreak

\section{\textsc{Primitives}}
\label{sec:primitives}
This section describes the implementation of major system primitives. These 
primitives are used by the processes for various types of functionality. All 
primitives are implemented as system calls which switch to kernel mode using a 
software interrupt.

\subsection{\textsc{Release Processor}}
The release processor primitive allows the operating system to make a
scheduling decision. Depending on its state, the currently running is
enqueued to either the \texttt{READY}, \texttt{BLOCKED\_MESSAGE}, or
\texttt{BLOCKED\_MEMORY} priority queue. Finally, the unblocked process
of highest priority is dequeued and a context switch is made to it. Pseudo code 
for this primitive is shown in Listing~\ref{releaseproclisting}.

\lstset{caption={Pseudo code for \texttt{release\_processor()}},
        label=releaseproclisting}
\begin{lstlisting}
int release_processor() {
    mode switch to kernel mode using system call exception

    if currently running process is not an i-process:
        enqueue currently running process to appropriate queue

    dequeue next process from highest non-empty priority queue

    context switch to dequeued process
    
    return RTX_SUCCESS
}
\end{lstlisting}

\subsection{\textsc{Set Process Priority}}

The set process priority primitive modifies the priority of a process
in the system. This primitive only accepts valid PIDs and priority
levels. If the updated process is unblocked and has a higher priority
running, then the running is preempted. Listing~\ref{setplisting}
contains pseudo code for this primitive.

\lstset{caption={Pseudo code for \texttt{set\_process\_priority(...)}},
        label=setplisting}
\begin{lstlisting}
int set_process_priority(int pid, int priority) {
    mode switch to kernel mode using system call exception
    
    if invalid PID or invalid priority:
        return RTX_ERROR
    
    if running process is the process to change:
        update running process to the new priority
    else:
        dequeue process from appropriate queue
        update priority of that process
        enqueue process to its new priority queue

    if the priority of the updated is greater than the running processes priority:
        call release processor

    return RTX_SUCCESS
}
\end{lstlisting}

\subsection{\textsc{Get Process Priority}}
The \texttt{get\_process\_priority()} primitive is used to retrieve the
priority of any process in the system. If an invalid PID is specified,
\texttt{RTX\_ERROR} is returned. Otherwise, the PCB is retrieved from
the process table and the priority of the process is returned. Pseudo
code for this primitive can be found in Listing~\ref{getplisting}.

\lstset{caption={Pseudo code for \texttt{get\_process\_priority(...)}},
        label=getplisting}
\begin{lstlisting}
int get_process_priority(int pid) {
    mode switch to kernel mode using system call exception
    
    if the pid is invalid:
        return RTX_ERROR

    get process from the process table using pid
    
    return process priority
}
\end{lstlisting}

\subsection{\textsc{Request Memory Block}}

The \texttt{request\_memory\_block()} is used by processes to
allocated one of the thirty two memory blocks in the system. All
processes have access to thirty of these blocks, while two are
reserved for usage by i-processes. If no memory blocks are available
to the running process, it becomes blocked on memory and is
preempted. Pseudo code for this primitive is in
Listing~\ref{requestmemory}.

\lstset{caption={Pseudo code for \texttt{request\_memory\_block()}}, label=requestmemory}
\begin{lstlisting}
void* request_memory_block() {
    mode switch to kernel mode using system call exception

    if running process is not an i-process:
        if there are no unreserved memory blocks free:
            set the state of the running process to blocked on memory
            call release processor
    else:
        if there are no memory blocks free:
            set the state of the running process to blocked on memory
            call release processor

    pop block of memory off the free list        
    get block index
    set corresponding bit in memory allocation field to 1

    return address of block
}
\end{lstlisting}

\subsection{\textsc{Release Memory Block}}

The primitive \texttt{release\_memory\_block()} is used by processes to
deallocate a previously allocated block of memory. If the block is invalid
\texttt{RTX\_ERROR} is returned. Once the block has been deallocated, if there
are processes blocked on memory, the highest priority blocked process is
unblocked and enqueued to the ready queue. If the priority of the unblocked
process is higher than the running process, the running process is preempted.
Pseudo code for this primitive can be found in Listing~\ref{releasememory}.

\lstset{caption={Pseudo code for \texttt{release\_memory\_block(...)}}, label=releasememory}
\begin{lstlisting}
int release_memory_block(void* memory_block) {
    mode switch to kernel mode using system call exception

    if memory block is not allocated or out of range:
        return RTX_ERROR
    else:
        push block back on to free list
        set correct bit in memory allocation field to 0

        if there are processes blocked on memory:
            set state of the highest priority blocked process to ready
            if priority of the unblocked is higher than the running process:
                 call release processor

        return RTX_SUCCESS        
}
\end{lstlisting}

\subsection{\textsc{Send Message}}

The \texttt{send\_message()} primitive is used to send a message to a
process in the system. Messages are enqueued onto a message queue in
the receiving process' PCB. If the receiving process is currently
blocked on messages, it is unblocked and enqueued to the ready
queue. If the receiving process is a higher priority than the running
process, the running is preempted. Listing~\ref{sendmsg} contains
pseudo code for this primitive.

\lstset{caption={Pseudo code for \texttt{send\_message(...)}}, label=sendmsg}
\begin{lstlisting}
int send_message(int pid, void* message_envelope) {
    mode switch to kernel mode using system call exception

    if pid is invalid:
        return RTX_ERROR

    update sender and receiver fields in message envelope
    enqueue message to message queue of receiving process

    if receiving process is blocked on messages:
        set state of receiving process to ready

    if priority of receiving process is higher than the running process:
        call release processor

    return RTX_SUCCESS
}
\end{lstlisting}

\subsection{\textsc{Receive Message}}

The \texttt{receive\_message()} primitive is used by processes to
receive a message from a process. If a message is available on the
running processes message queue, it is immediately dequeued and
returned. Otherwise, the state of the running process is set to
blocked and it is preempted. Pseudo code for this primitive can be
found in Listing~\ref{receivemsg}.

\lstset{caption={Pseudo code for \texttt{receive\_message(...)}}, label=receivemsg}
\begin{lstlisting}
void* receive_message(int* sender_id) {
    mode switch to kernel mode using system call exception

    while message queue for running process is empty:
        set state of running process to blocked on messages
        call release processor

    dequeue message from message queue of running process
    set sender_id parameter to sender pid found in message envelope
    return message
}
\end{lstlisting}

\subsection{\textsc{Delayed Send}}

The \texttt{delayed\_send()} primitive allows processes to send a
message which will not be delivered until a specified time has
elapsed. The current system time and the delay amount are stored in
the message envelope. The message is then inserted into the delayed
message queue; this queue is ordered such the next message to be sent is at
the front. There is only one queue for delayed messages, which is shared
between all processes. Sending of the delayed message is handled by
the timer i-process. Pseudo code for this primitive can be found in
Listing~\ref{delay}.

\lstset{caption={Pseudo code for \texttt{delayed\_send(...)} }, label=delay}
\begin{lstlisting}
int delayed_send(int process_id, void* message_envelope, int delay) {
    switch to kernel mode using system call exception

    if receiver pid or delay value is invalid:
        return RTX_ERROR
    else:
        set sender id field of message_envelope
        set receiver id field of message_envelope
        set the delay amount field of message_envelope
        set the send time field of message_envelope
        insert into delayed messages queue

        return RTX_SUCCESS
}
\end{lstlisting}


\pagebreak 

\section{\textsc{Processes}}
\label{sec:processes}
The operating system consists of several system, user, interrupt, and test 
processes. These processes make up the main functionality of the operating 
system. In order to be implemented properly these processes make use of the 
primitives described above. Processes are assigned different priority levels 
which are used to enforce what process should be running.
\subsection{\textsc{Null Process}}
The process method is defined as an infinite loop that constantly calls 
\texttt{release\_processor()}. It has a priority of four and a process ID of 
zero. This process is outlined in Listing~\ref{nullplisting}.

\lstset{caption={Pseudo code for the null process},
        label=nullplisting}
\begin{lstlisting}
void process_null() {
        repeatedly call release processor
}
\end{lstlisting}

\subsection{\textsc{KCD Process}}
The \texttt{process\_kcd()} handles keyboard input and delegation of keyboard 
commands. Processes can register for commands by sending the 
\texttt{process\_kcd()} a message of a specific type. If a keyboard input 
message is received and a matching command is found the message is forwarded to 
the registered process. The pseudo code for the KCD can been seen in 
Listing~\ref{kcd}.

\lstset{caption={Pseudo code for KCD process}, label=kcd}
\begin{lstlisting}
void process_kcd() {
    in an infinite loop:
        receive a message
        if the message type is a keyboard input message:
            if the message body is a registed command:
                send message to registered process
            else:
                release memory block of message
        else if it is a command registration message:
            save the PID and command to be registered
        else:
            release memory block of message
}
\end{lstlisting}

\subsection{\textsc{CRT Process}}
The \texttt{process\_crt\_display()} blocks until a message is
received. If the type is correct, then the for each character in the message 
data it is sent to the UART shared memory and UART transmit interrupt is 
triggered. Then, the memory of the message is released and release processor is 
called. Pseudo code for the process can be found in Listing~\ref{crt}.

\lstset{caption={Pseudo code for CRT display process}, label=crt}
\begin{lstlisting}
void process_crt_display() {
    in an infinite loop:
        receive a message
        if message type is keyboard input or output:
            for each character in the message body:
                write character to uart shared memory
                trigger a uart transmit interrupt
        release memory block of message
        call release processor
}
\end{lstlisting}

\subsection{\textsc{UART I-Process}}

The UART i-process is responsible for handling all UART interrupts,
both receive ready and transmit ready. When the process is run, it
checks which of these interrupts fired. If the interrupt was receive
ready, the input character is appended to a string buffer. When a
keyboard command is detected, it is forwarded in a message to the KCD
process. When a hot key command is detected, the relevant hot key
command is run. When a transmit interrupt is received, a character is
printed out the UART from a location in shared memory. Pseudo code for
this process can be found in Listing~\ref{uart}.

\lstset{caption={Pseudo code for the UART I-Process}, label=uart}
\begin{lstlisting}
void i_process_uart() {
    in an infinite loop:
        determine if the interrupt was receive or transmit

        if the interrupt is receive ready
            read in character

            if the character was a null
                restart loop

            append character to string buffer
            
            if the character was a carriage return:
                append null to string buffer
                
                if the first character in the buffer is '%':
                    request a memory block for a new message
                    copy string buffer into message data field
                    set message type to keyboard input
                    send message to KCD
                else if the first character in the buffer is '!':
                    call uart debug decoder function with string buffer        

            else:
                write the character to uart shared memory
                echo the character by triggering a transmit interrupt

        else if the interrupt is transmit ready:
            write character from uart shared memory to serial port
            mask transmit ready interrupt
            
            if the character is a carriage return:
                print carriage return
                write newline character to uart shared memory
                unmask transmit ready interrupt
            
        release processor
}
\end{lstlisting}

\subsection{\textsc{Timer I-Process}}

The \texttt{i\_process\_timer()} process runs in response to timer
interrupts, specifically the counter overflow interrupt. Internally,
the process keeps track of the number of milliseconds elapsed since
its start. Whenever the process runs, it is known that a millisecond
has elapsed and the system clock is incremented. Next, any delayed
messages that are ready to be sent are forwarded to the specified
receiver processes. Pseudo code for this process can be found in
Listing~\ref{timer}.

\lstset{caption={Pseudo code for the timer I-Process}, label=timer}
\begin{lstlisting}
void i_process_timer() {
    set system clock to 0

    in an infinite loop:
        call release processor
        increment system clock

        if delayed messages queue is not empty:
            while the first message in the queue is ready to be sent:
                dequeue the first message
                send message to specified receiver        
}
\end{lstlisting}

\subsection{\textsc{Wall Clock Process}}
The \texttt{process\_wall\_clock()} is used to display the current
system time on the CRT. The clock is given an initial time when the
\texttt{\%WS} command is received and is updated every second. The
clock is printed until the \texttt{\%WT} command is received.

The clock is updated by repeatedly sending itself messages delayed by
one second. When the clock reaches a time of 24:00:00 it is reset to
00:00:00. Pseudo code for the wall clock process can be found in
Listing~\ref{wallclock}.

\lstset{caption={Pseudo code for the wall clock process}, label=wallclock}
\begin{lstlisting}
void process_wall_clock() {
    set the clock to 0

    in an infinite loop:
        receive message

        if the message sender id is the wall clock pid:
            increment clock

            if the clock value is 24 hours:
                reset clock to 0

            if the clock is currently running:
                send our self a delayed message with a delay of one second
                convert clock to hours, minutes and seconds
                convert hours, minutes, seconds to time string
                send time string as message to CRT

        else if the message sender id is the KCD pid:
            if the command was '%WS':
                if the clock is currently off:
                    send our self a delayed message with a delay of one second

                parse string and validate input
                set the clock to corresponding input value
                turn clock on
            else if the command was '%WT':
                turn off clock

        release memory block of message
}
\end{lstlisting}

\subsection{\textsc{Set Priority Command Process}}

The \texttt{process\_set\_priority\_command()} process allows a user
to change the priority of any process in the system directly from
terminal. The process registers the '\%C' keyboard command which takes
the form '\%C pid priority'. When this command is received, the
process parses the arguments, validates them, and if they are correct
it changes the priority of the selected process. Pseudo code for this
process can be found in Listing~\ref{lst_set_prio_cmd_proc}.

\lstset{caption={Pseudo code for the set priority command process}, 
        label=lst_set_prio_cmd_proc}

\begin{lstlisting}
void process_set_priority_command() {
    register '%C' command with KCD process
    
    in an infinite loop:
        receive a message

        skip '%C' in message

        parse process id
        validate process id

        parse priority level
        validate priority level

        validate line endings
        
        call set_process_priority with the provided arguments

        release memory block
        release processor
}
\end{lstlisting}

\subsection{\textsc{Test Processes A, B, C}}
The system also has three test processes implemented. These processes are used
to stress test our system by depleting memory blocks. \texttt{Process A}
requests a memory block and on receiving one sends a message to \texttt{Process
B}. \texttt{Process B} receives the message and then sends the message to
\texttt{Process C}. \texttt{Process C} then gets a message and checks if the
number of messages it has received is divisible by 20. If it is it prints out
"Process C" to the UART1. Then the process hibernates for 10 seconds. All of
these processes loop forever in order to deplete memory. In order to have these
processes work properly within our system \texttt{Process C} was assigned a
priority of 1 and the other two processes were assigned a priority of 2.

\pagebreak

\section{\textsc{Software Interrupt Handler}}
\label{sec:soft_isr}

The system uses a single software interrupt handler for performing context
switches into kernel mode for kernel functions. It is initialized in the 0th
location on the trap table. The trap is only called from the function
\texttt{do\_system\_call(int call\_id, int* args, int num\_args)} which is used
to abstract common code used by all the primitives. The software handler itself
is called by the trap mnemonic is \texttt{system\_call()} which retrieves
the parameters from data registers 1 to 4. It calls the appropriate kernel
level function based on the call id parameter. The call ids are defined in an
enumerator. 

The software interrupt handler, as described in Listing~\ref{lst_system_call},
is installed in vector 0 of the trap table. The operation of this interrupt
handler retrieves the parameters from the data registers, saves the rest of
the user process state into the stack, and calls the appropriate kernel level
function. The parameters that have been retrieved are passed in as arguments for 
the function. The function return value is stored in one of the data registers 
that is to be passed back to the function that called the trap. Finally, the 
function returns from exception to switch back from kernel mode into user mode.

\lstset{caption={Software interrupt handler},
        label=lst_system_call}
\begin{lstlisting}
void system_call() {
    disable interrupts

    retrieve call id and arguments from data registers

    save current user process state

    use call id to select the appropriate kernel level function 
    pass in the arguments as parameters to the function

    restore user process state

    load return value into a data register

    return from exception
}
\end{lstlisting}

\pagebreak
\section{\textsc{Hardware Interrupt Handlers}}
\label{sec:hard_isr}
% data structures, functional outline, pseudocode
Two hardware interrupt handlers were implemented. The first is for the timer
interrupts and the second is for the UART for keyboard input from the user.
They both use very similar logic. All of the data and address registers are
moved onto the stack after interrupts are disabled. Then the processor is
preempted with the the respective interrupt process selected as the next
running process. In the case of the timer interrupt, the interrupt event is
acknowledge to prevent the interrupt from firing multiple times for the same
event. After the interrupt process releases the processor, and the processor
returns back to the interrupted process, the data and address registers are
restored from the stack and the interrupt handler returns from the exception.
See Listing~\ref{listing_hwi_handler} for the pseudo code description of the
two interrupt handlers.

\lstset{caption={Hardware interrupt handler},
        label=listing_hwi_handler}
\begin{lstlisting}
void interrupt_handler() {
    disable interrupts
    move data and address registers onto the stack

    acknowledge the interrupt
    preempt processor with the appropriate interrupt process

    restore data and address registers from the stack

    return from exception
}
\end{lstlisting}

Interrupt service routines are loaded into the interrupt vector table and
initialized with the required settings in the initialization period of the
system start up. This is described in Section~\ref{sec:rtx_init}.

\pagebreak

\section{\textsc{Hot Keys}}
\label{sec:hot_keys}
% What they are, their functionality, pseudocode 

A total of five hot key commands were implemented to assist in debugging and
testing of the system and user processes. A hot key decoder function was used to
interpret the inputted command from the UART interrupt process. The interrupt
process checks the first character for an exclamation mark ('!') and calls the
hot key function directly with the input string as a parameter. See
Table~\ref{hot_keys_table} for a summary of the hot key strings and
descriptions. All of the code related to the debug hot keys are wrapped in
\texttt{\#ifdef \_DEBUG\_HOTKEYS} and \texttt{\#endif} to ensure that during 
normal operation of the system that this code will not interfere.

\begin{table}[h]
    \caption{Hot Keys and Descriptions}
    \label{hot_keys_table}
    \begin{tabular}{| r | r |}
        \hline
        Hot Key Command & Description \\
        \hline
        !RQ & Print ready processes and priorities \\ 
        !BMQ & Print processes blocked on memory \\ 
        !BRQ & Print processes blocked on receiving messages \\ 
        !FM & Print the current number of free memory blocks \\ 
        !M & Print the last $i$ messages sent where $i$ is the debug message
        log size \\
        \hline
    \end{tabular}
\end{table}

Each of the debugging functions are kernel level calls as they require access
to data structures in the kernel. This is done using the same soft interrupt
method as the the rest of the kernel routines. The hot key command decoder
uses a similar parsing method as the \texttt{process\_wall\_clock()} process to
interrupt which kernel routine to run. The debugging processes related to the
ready queues and blocked queues use the kernel print queue function as
described in Listing~\ref{lsting_k_print_queue}. The specific queue debugging 
function simply passes in the relevant queue into the queue print function.

\lstset{caption={Kernel print queue pseudo code}, 
        label=lsting_k_print_queue}
\begin{lstlisting}
int queue_debug_print(process_queue queue[]) {
    for each process in the queue:
        print process ID
        print process priority
}
\end{lstlisting}

The hot key, \texttt{!FM}, simply outputs the number of free blocks and the
allocation bit field. The allocation bit field is described in
Section~\ref{sec:mem_management}.

The last hot key, \texttt{!M}, is the most complicated of the five. In each of
the kernel functions related to message sending and receiving, a copy of each
message is stored on a circular buffer that contains a data structure that
maintains a record of the sender ID, receiver ID, message type, the first 16
bytes of data, and the time stamp of the message. Separate circular buffers are
used for the sent messages and received messages to assist in debugging if a
message has been sent but not received.  When the hot key is called, it goes
through every message in both queues and outputs first all the sent messages
and then all of the received messages. The number of messages store in the two
queues can be changed by modifying the \texttt{DEBUG\_MESSAGE\_LOG\_SIZE}
definition. 

\pagebreak

\section{\textsc{Initialization}}
\label{sec:rtx_init}
The RTX initialization happens in four phases: process, priority queues, 
interrupts, and the memory management system. Pseudo code for everything except
memory management can be seen in Listing~\ref{initlisting}.

The process initialization occurs in \texttt{init\_processes()} and is done in
two stages. The first is an initialization of the test processes. Second is the
initialization of the null process and the user processes. For each process, a
PCB is created on the process table. This PCB is then added to the kernel's own
process list but is not yet added to a priority queue. For each process in
the kernel's process list a stack pointer is created and an exception frame is
added to this new stack. The format/vector word portion of the exception frame
is set to \texttt{0x4000} to represent a 4 byte aligned stack and the status register
portion is \texttt{0x0000}. In the frame, the program counter is assigned to the entry
point of the process.

Next, priority queues are initialized. The data structures are already
allocated but not initialized. To ensure proper operation, the queues are
put into a consistent empty state. After the queues are setup, the processes
are added to appropriate queue.

Then interrupts are setup. First, the vector base register (VBR) for the
interrupts is set to memory location \texttt{0x10000000}. Next, the soft interrupt for
\texttt{system\_call()} is installed to the VBR at vector 0 as outlined in
Section~\ref{sec:soft_isr}.

\lstset{caption={Pseudo code for RTX initialization},label=initlisting}
\begin{lstlisting}
void init() {
    init_processes()
    init_priority_queues()
    init_interrupts()
}

void init_processes() {
    init_test_processes()
    init_user_processes()
    for each process in process list:
        initialize stack pointer
        push PC portion of exception frame onto the stack
        push F/V and SR portions of exception frame onto the stack
}

void init_test_processes() {
    for each test process: 
        setup PCB
        add to process list
}

void init_user_processes() {
    user process list:
        null process
        Process A
        Process B
        Process C
        Uart
        Timer
        CRT
        KCD
        Wall Clock Display
        Set Priority Process

    for each process in user process list:
        setup PCB
        add to process list
}

void init_priority_queues() {
    for each process: 
        add process to correct priority queue
}

void init_interrupts() {
    initialize VBR to 0x10000000
    install system call ISR at vector 0
}
\end{lstlisting}

The final phase of the RTX initialization is memory management.  Memory blocks
are allocated beginning at the start of free memory as defined by the linker.
The function call \texttt{init\_memory(void* memory\_start)}, as seen in
Listing~\ref{meminit}. A free list is used to implement linking and management
of the memory blocks. Over-allocation and double deallocation prevention is
handled by a bit field containing a bit for each memory block. When the memory
is allocated the associated bit is set to 1 and for when it is released it is
set back to 0.

\lstset{caption={Pseudo code for RTX initialization of memory
        management},label=meminit}
\begin{lstlisting}
void init_memory(void* memory_start){
    
    head of memory = start of free memory + (number of memory blocks - 1)*(size of memory block)
    
    for each block in free list:
        store address of next free memory block in current block

    initialize the memory allocation bit field to 0
}
\end{lstlisting}

\pagebreak 

\section{\textsc{Implementation}}
\label{sec:implementation}
This section is responsible for outlining some basic implementation techniques 
regarding the testing of our operating system and the measurement of certain 
primitives. It also includes a basic discussion of the role of each group member. 
 
\subsection{\textsc{Responsibilities}}
The majority of the kernel was written as the whole group. The queues, context
switching, and most of the primitives were designed and developed together to 
ensure that every group member would have a solid understanding of the operation 
of the system. Memory management and message handling was also designed in this
manner. As a result, there were less bugs not only in the primitives but also
in the processes implemented later on by individual members. For the most part,
the debugging of the integration of processes and primitives was also handled as 
a group.

\subsubsection{\textsc{Ben Ridder}}
The majority of the processes were implemented by Ben Ridder. He wrote the
hardware interrupt and interrupt process for UART1 paired with the two system
processes, KCD and CRT display, that are associated with it. Hot keys was also
a part of this, with additional changes made to the kernel to create the
desired functionality of the hot keys described in Section~\ref{sec:hot_keys}.
The other system process implemented by Ben was the set priority command
process. User processes A, B, and C were written by Ben as described in the
project documentation. Finally, he wrote boilerplate prior to the next group
session to limit the amount of time spent as a group on it.

\subsubsection{\textsc{Casey Banner}}

Implementation of the core system primitives, mode switching,
scheduler, and context switching were Casey Banner's
responsibility. He implemented the system call infrastructure, as well
as several debugging print routines. Necessary standard library
functions such as string formatting, memory copying, and character
parsing functions were implemented by Casey. Most of the
implementation was done while pair programming with other members of
the group. Casey also handled implementation of the test case processes
which were designed by David. Additionally, Casey handled setting up
the version control system as well as the initial Makefile structure.

\subsubsection{\textsc{David Janssen}}
It was David's responsibility to write the documentation for each part of the 
project. Also, he worked in a pair programming environment to assist with the 
development of the memory management system, the request and release memory 
block primitives, the get and set priority primitives, and the wall clock 
process. Finally,  David had to come up with test cases that would be 
implemented to verify the functionality of each primitive.   

\subsection{\textsc{Test Plan}}
Testing for the operating system was done in a consistent fashion throughout all
parts of the project. After coding a new process or primitive a new test case 
was written in order to verify its functionality. This test case was then 
implemented using a set of processes which would in turn run these tests. This 
would include tests specifically for memory, messages, the changing of priority, 
etc. Once it was determined that a new addition to the operating was running as 
per specification, the entire test suite was run to verify no negative impact on 
the system. 

To ease the task of debugging, a very basic \texttt{printf} function 
was created which allowed decimal and hexadecimal data to be displayed. This 
was very useful as it allowed us to print the contents of certain registers and 
variables when we were unable easily discover issues when a test case failed or 
functionality was not as expected. 

This testing strategy was not used in part one of the project as it had a 
separate test program written to test its functionality. The user process A, B, 
and C were used to stress test our operating system and verify its behaviour 
under load. 

\subsection{\textsc{Measurement Plan}}
\label{measurementplan}

Three methods were considered for profiling the system call
primitives: a stopwatch, a system millisecond timer, and using a second
hardware timer. In each of these cases, the primitive was run several
thousand times in a loop in order to find an average execution
time. In order to avoid blocking during the test run, any relevant
cleanup was done within the loop, but was excluded from the
measurement. For example, when \texttt{request\_memory\_block()} was
profiled, the memory block was immediately released; however, the time
to release the memory block was not added to the total.

The first method involved manually timing the execution of ten
thousand runs of a particular primitive. A message was printed at the
beginning and the end of the loop and stopwatch was used to measure
the total time elapsed. This method proved ineffective due to inherent
delays in printing and human response time.

Second, the system millisecond timer was used to measure the execution
time of the function. This method was not effective because the
millisecond timer did not provide enough granularity in order to get
an accurate result; the average run time for all primitives was less than a
millisecond.

Finally, the second hardware timer was used to measure execution time
of the primitives. By setting the timer to increment its internal
counter at a rate of 180kHz, this profiling method was able to achieve
5.555 microsecond granularity. Using this timer a process was
developed that reset the timer counter register before running a
primitive, and checked its value after the primitive finished
executing. In this fashion the primitive execution time could be
accurately measured. Due to the shortcomings of the first two methods,
this method was used to complete the measurements.

\chapter{\textsc{Major Design Changes}}
All in all, there were only two design changes that needed to be done. The
first was related to the initial implementation of the scheduling system. The
other was how the processes managed their memory blocks as allocated by the
system and preventing key interrupt processes from blocking on whilst waiting
for memory.

\section{\textsc{Done Queues}}
Initially, the priority queuing system was implemented using several different
queues for various process states and this included a done queue. The purpose
behind the done queue was to ensure that every process would execute in order
from highest to lowest priority. As a result of this system, the demo test
cases failed due to differing expectations of how the scheduling system
operates. These were not clear in the documentation but was clarified at the
demo.

After the demo, the system was changed such that the scheduler would only run
processes in the highest non-empty priority queue. In hindsight, the previous
scheduler would have been ineffective in the requirements of the system for the
last portion of the project. It would have worked better in an environment that
employed time slicing to select the next process periodically.

\section{\textsc{Memory Management}}
Two changes where made to both the implementation of memory and how the
processes used memory blocks. The stress test processes A, B, and C clearly
indicated issues with the original memory system. To ensure consistency with
the interrupt processes, two memory blocks were set aside just for the UART and
timer processes. Memory management within the system processes also became an
important consideration.

Reserving a couple blocks of memory just for the interrupt processes limited
how often the two interrupt processes became blocked on memory. Blocking the
UART on memory will result in the system appearing like it locked up since
nothing could be inputted from the keyboard or outputted to the display.
Preventing this process from blocking increases the responsiveness of the
system.

From the perspective of the system processes, the initial usage of memory blocks was
naively assuming that a memory block could be requested whenever needed without
worrying about blocking. This was not the case. As a result of this, system processes
would block and create deadlocks for a new memory block as often the process
already had a memory block allocated to it. By reusing memory blocks, processes
would deadlock less often and memory waste was reduced.

\chapter{\textsc{Measurements}}

Three primitives were measured for time of execution. Kernel primitives
\texttt{receive\_message()}, \texttt{send\_message()}, and
\texttt{request\_memory\_block()} were all profiled. The purpose of this
profiling is to ensure that these key kernel level functions perform quickly,
thus ensuring a responsive system.

\section{\textsc{Measurement Data}}

Using the hardware timer method described in
Section~\ref{measurementplan} execution times for the
\texttt{receive\_message()}, \texttt{send\_message()}, and
\texttt{request\_memory\_block()} primitives were collected. This data
can be found in Table~\ref{measurementdata}.

\begin{table}[H]
    \caption{Execution time of primitives}
    \label{measurementdata}
    \center{
        \begin{tabular}{| l | c |}
          \hline
          {\bf Primitive Name} & {\bf Average Execution Time ($\mu$s)} \\
          \hline
          \texttt{receive\_message()} & 244.44 \\
          \hline
          \texttt{send\_message()} & 255.55 \\
          \hline
          \texttt{request\_memory\_block()} & 327.77 \\
          \hline
        \end{tabular}
    }
\end{table}

\section{\textsc{Measurement Analysis}}

In general, the execution time for the primitives was on the order of
hundreds of microseconds. Given that these primitives perform system
functions which are called often, it is important that they are responsive.

Of the three primitives that were profiled,
\texttt{receive\_message()} was the fastest. When the running is not
blocked, all that this primitive is required to do is dequeue a
message, update the sender ID value, and then return. Dequeuing a
message is a simple operation, and updating the sender ID is trivial.

Next is the \texttt{send\_message()} primitive. It is very similar in
complexity to the \texttt{receive\_message()} primitive; however, it
performs the additional steps of looking up the receiving process'
PCB, and updating fields on the message envelope. Additional, it needs
to check if a process was unblocked and possibly preempt the running
process.

Finally, \texttt{request\_memory\_block()} is the slowest of the three
primitives profiled. The increased execution time can be attributed to
the math that is required to calculate the allocated block's index in
the memory allocation field. Also, when checking to see if there is
any free memory blocks, the number of set bits in the memory
allocation field need to be counted. This operation is expensive; 32
iterations involving a bit shift need to be performed.

\chapter{\textsc{Lessons Learned}}
Several lessons were learned as a result of undertaking this project as a group
of three. In hindsight, there are several areas in which we could improve both
technically and organizationally. On the technical side, the majority of these
areas were related to the processes. Organizationally, we excelled as a result
of planning and time management.

\section{\textsc{Things to Improve}}

\subsection{\textsc{Technical}}

Overall, we were very pleased with our implementation of the operating system.
However, during the part four lab demo, three issues were highlighted which
should be fixed if future revision was to be made. First, it was discovered
that some user processes are running in supervisor mode. Second, the current
UART I-process implementation has a possibility of blocking on memory. Third, our
check for the number of memory allocated runs in $O(n)$ time, where $n$ is the
total number of memory blocks in the system.

During the lab demo, a privileged instruction was inserted into a process that
should have been running in user mode. When executed, the instruction did not
raise a privileged instruction exception as it should have. This is due to the
user process erroneously running in supervisor mode. The suspected cause for
this is an incorrect initial exception on the user processes stack.

The UART I-process contains a call to \texttt{request\_memory\_block()} inside
the code path that handles keyboard commands that begin with '\%'. If the
system is starved for memory and a user enters such a keyboard command, the
I-process will be come blocked. Obviously, this is not desired as this will
cause all user input to block which will in turn cause hot keys to no longer
function.

Our memory manager implementation includes an allocation bit field which needs
to be checked each time a memory block is requested in order to determine how
many blocks are free. This check could be improved to run in $O(1)$ time by
keeping a count of the allocated memory blocks alongside the bit field.

Finally, there are some implementation details that could be improved upon.
Shared memory is used for IPC between the CRT process and the UART I-process
when message passing would likely be a more robust choice. 

\subsection{\textsc{Organization}}

Initially we used issue tracking software to manage tasks for entire project
including documentation, programming, and project milestones. However, after
the part two demo we stopped adding new issues to the tracking software and
neglected it. This was due to the fact that we were meeting quite often and no
longer felt the need to formally log all the issues. This did not have a large
negative on our groups success with project. Despite this, we were able to
complete all the milestones on time and deliver part 4 ahead of the deadline.
However, if we had not been meeting in person frequently, it is likely that
productivity would not have been the same. If we were required to do this
again, issue tracking software would be a critical tool in informing members of
the group that could not attend meetings about the project's current status.

\section{\textsc{Things Done Well}}

\subsection{\textsc{Technical}}

Two particular parts of our operating system were implemented in a particular
effective way. First of all, the memory manager is particularly robust.
Secondly, great effort was used to ensure as many operations would occur in 
$O(1)$ time as possible.

The memory manager implementation uses an allocation bit field to keep track of
which memory blocks have been allocated. This is used during deallocation to
prevent against double deallocation and deallocating blocks that have not been
allocated. The check for double deallocation runs in $O(1)$ time. Also, the
memory manager includes out of range checks to avoid deallocating invalid
blocks of memory.

With the exception of counting the allocated memory blocks, inserting messages
into the delayed message queue, and checking which process is first in a
priority queue, all system functions run in $O(1)$ time. Due to the design of
the data structures in the operating system, operations such as removing a
process from the middle of a queue do not need to search the queue for the
process. Additionally, the design decision to use a free list for our memory 
management system, instead of a basic linked list, was made so that finding the 
next free memory block could be done without having to traverse the whole list. 
The free list also saves memory, as a counter or pointer could be kept to the 
next available memory block, but with a free list this is not necessary as the 
block it self stores the next pointer.

\subsection{\textsc{Organization}}

Our time management skills were very effective throughout this project. We had
three extremely busy schedules which we had to coordinate in order to work on
the project as a group. Due to our planning of group work sessions far ahead of
the milestone deadlines, we were to complete the required work without any late
night lab sessions. Because of our organizational skills, we did not require
access to the hardware near the due date of the final submission. This was
fortuitous, as access to a coldfire board became extremely difficult as the due
date approached.

Also, during the course of the project, no major rewrites or architectural
changes were required. This was due to the fact that significant planning was
done before any code was written. By having group discussions and utilizing
pair programming techniques, code quality was high and human error was
minimized.

\end{document}
