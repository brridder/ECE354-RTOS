\documentclass[oneside]{article}
\usepackage{listings}
\usepackage[hmargin=1.5in,vmargin=1.0in]{geometry}
\usepackage{graphicx}
\usepackage[T1]{fontenc}
\usepackage[latin1]{inputenc}
\usepackage{tikz}
\usetikzlibrary{shapes,arrows}

\begin{document}
\lstset{language=C, 
        frame=single, 
        breaklines=true,
        basicstyle=\small\ttfamily,
        columns=fullflexible}

\tikzstyle{decision} = [diamond, draw, fill=blue!20, text width=4.5em, text
    badly centered, inner sep=0pt]
\tikzstyle{block} = [rectangle, draw, fill=blue!20, text width=5em, text
    centered, rounded corners, minimum height=4em]
\tikzstyle{line} = [draw, -latex']
\tikzstyle{cloud} = [draw, ellipse, fill=red!20, node distance=3cm, minimum
    height=2em]

\section*{ECE 354: Part 4}
Group: ECE.354.S11-G031 \\
Members: Ben Ridder (brridder), Casey Banner (cccbanne), 
David Janssen (dajjanss) \\ \\
This document is concerned with the high level design of our six testing 
processes. These processes are used by three test cases to check the 
validity of different parts of the operating system. The first test case is 
dedicated to testing the functionality of the delayed send primitive. Next, the 
memory management system is tested with regards to the blocked queue and 
allocation. Last, the third test case examines multiple situations where the 
operating system is expected to return an error. The final process is used as a 
test management system that tracks the success and failures of the prior test 
cases which inherently tests message passing. 

\subsection*{Delayed Send Tests}
This test case makes use of two processes: \texttt{test\_delay\_sender()} and 
\texttt{test\_delay\_receiver()}. The first process sends six delayed messages 
with unique delays to the receiving process. These messages are not sent in the 
order they are expected to be received in. Next, the receiver process runs and 
attempts to receive these messages in the correct order. If all the messages 
are received in the correct order a \texttt{TEST\_SUCCESS} message is sent to 
the test management process, otherwise a \texttt{TEST\_FAILURE} message is sent.
 If there is a problem with the message sending of the \texttt{delayed\_send()} 
primitive the testing harness will hang.

\subsection*{Memory Management Tests}
This test uses two processes, \texttt{test\_memory\_watchdog()} and 
\texttt{test\_memory\_allocator()}, to verify the functionality of blocked 
queues and memory allocation. First, the watchdog process allocates a block and 
then sends a message to the allocator process. Upon receiving this message the 
allocator requests as many memory block as are available. Once it is blocked the 
watchdog process will continue to run. The next step of the watchdog process is 
to release the memory block that it requested, and then release itself. At this 
point the allocator process becomes unblocked and is now able to release all of 
its memory blocks. If no deallocations fail then \texttt{TEST\_SUCCESS} is sent 
to the the test management system, else \texttt{TEST\_FAILURE} is sent.

\subsection*{Error Checking Tests}
Error checking tests deal with the edge cases that should return error values
from the kernel primitives. Like all the other tests, this is pseudo-scheduled
from the managing process using differing priority levels and blocking on a
message. The first few checks are against the memory system. It checks that
releasing memory that is in locations both over and under what the memory
blocks are located. Next, it tries to deallocate the same memory block twice.
For each primitive that takes a process id as a parameter, an id that is out
of range is sent in to ensure that the ranges are checked by the primative. The
last error check performed is for the delayed send to ensure that a negative
delay time is handled.

\subsection*{Test Management System}
The three tests are ultimately controlled by the last of the six processes,
which is run right after the first test is completed. Once each set of tests is
completed, the manager process receives a message from the testing processes.
The next test set is selected by lowering the finished processes' priorities and
raising the priorities of the next test processes. Results of each test are
recorded for the final output of results. Once all tests are completed, the
results are outputted in the required format. This management process should
not be preempted by the other test processes due to it having a higher priority
level then the others. Test processes are blocked waiting for a message from
the test manager. This message block is passed back and forth from the test
processes and the manager so it is ultimately released by the manager.
Listing~\ref{lsttestmansys} provides an outline of this process.

\lstset{caption={Test Management System Pseudo Code}, label=lsttestmansys}
\begin{lstlisting}
void test_management():
    while(1):
        receive message
        if message is not from the last test case:
            lower priority of previous test processes
            raise priority of next test processes
            send message to start of next test process
        if message is from last test case:
            release message block
        check message for test result:
            if successful:
                increment success counter
            if failure:
                increment failure counter
            output result
        
        if all tests are done:
            output total results
\end{lstlisting}
\end{document}
