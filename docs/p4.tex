\documentclass[oneside]{article}
\usepackage{listings}
\usepackage[hmargin=1.5in,vmargin=1.0in]{geometry}
\usepackage{graphicx}
\usepackage[T1]{fontenc}
\begin{document}
\lstset{language=C, 
        frame=single, 
        breaklines=true,
        basicstyle=\small\ttfamily,
        columns=fullflexible}
\section*{ECE 354: Part 4}
Group: ECE.354.S11-G031 \\
Members: Ben Ridder (brridder), Casey Banner (cccbanne), 
David Janssen (dajjanss) \\ \\
This document is concerned with the high level design of our six testing 
processes. These processes are used by three test cases to check the 
validity of different parts of the operating system. The first test case is 
dedicated to testing the functionality of the delayed send primitive. Next, the 
memory management system is tested with regards to the blocked queue and 
allocation. Last, the third test case examines multiple situations where the 
operating system is expected to return an error. The final process is used as a 
test management system that tracks the success and failures of the prior test 
cases which inherently tests message passing. 

\subsection*{Delayed Send Tests}
This test case makes use of two processes: \texttt{test\_delay\_sender()} and 
\texttt{test\_delay\_receiver()}. The first process sends six delayed messages 
with unique delays to the receiving process. These messages are not sent in the 
order they are expected to be received in. Next, the receiver process runs and 
attempts to receive these messages in the correct order. If all the messages 
are received in the correct order a \texttt{TEST\_SUCCESS} message is sent to 
the test management process, otherwise a \texttt{TEST\_FAILURE} message is sent.
 If there is a problem with the message sending of the \texttt{delayed\_send()} 
primitive the testing harness will hang.

\subsection*{Memory Management Tests}
This test uses two processes, \texttt{test\_memory\_watchdog()} and 
\texttt{test\_memory\_allocator()}, to verify the functionality of blocked 
queues and memory allocation. First, the watchdog process allocates a block and 
then sends a message to the allocator process. Upon receiving this message the 
allocator requests as many memory block as are available. Once it is blocked the 
watchdog process will continue to run. The next step of the watchdog process is 
to release the memory block that it requested, and then release itself. At this 
point the allocator process becomes unblocked and is now able to release all of 
its memory blocks. If no deallocations fail then \texttt{TEST\_SUCCESS} is sent 
to the the test management system, else \texttt{TEST\_FAILURE} is sent.

\subsection*{Error Checking Tests}

\subsection*{Test Management System}

\end{document}
